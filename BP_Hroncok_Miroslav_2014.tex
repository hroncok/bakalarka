% options:
% thesis=B bachelor's thesis
% thesis=M master's thesis
% czech thesis in Czech language
% english thesis in English language
% hidelinks remove colour boxes around hyperlinks

% arara: xelatex: { shell: yes }
% arara: makeglossaries
% arara: biber
% arara: xelatex: { shell: yes }
\documentclass[thesis=B,czech]{template/FITthesisXE}

\usepackage{ graphicx }		% graphics files inclusion
\usepackage{ dirtree } 		% directory tree visualisation
\usepackage{ longtable } 	% tables which Pandoc use

\addbibresource{library.bib}

% make list of acronyms
\makeglossaries
\input{acronyms.tex}
\glsaddall	% add even unused acronyms

% % % % % % % % % % % % % % % % % % % % % % % % % % % % % % 

\acknowledgements{\input{podekovani.tex}}
\abstractCS{\input{abstrakt_cs.tex}}
\abstractEN{\input{abstrakt_en.tex}}
\input{meta.tex}

\begin{document}

\begin{introduction}
\input{Uvod.tex}
\end{introduction}

\input{Doporuceni-pro-REST.tex}
\input{Pozadavky.tex}
\input{Navrh-API.tex}
\input{Realizace.tex}

\begin{conclusion}
\input{Zaver.tex}
\end{conclusion}

\printbibliography[title={Zdroje}]

\appendix

\chapter{Seznam použitých zkratek}
\printglossary[type=\acronymtype,style=acronyms]

\chapter{Obsah přiloženého média}

\begin{figure}%
	\dirtree{%
		.1 doc\DTcomment{přiložené dokumenty}.
		.1 readme.md\DTcomment{stručný popis obsahu média}.
		.1 src.\DTcomment{~}.
		.2 impl\DTcomment{zdrojové kódy implementace}.
		.2 thesis\DTcomment{zdrojová forma práce ve formátu Markdown a \LaTeX{}}.
		.1 thesis.pdf\DTcomment{text práce ve formátu PDF}.
	}
\end{figure}

\end{document}
